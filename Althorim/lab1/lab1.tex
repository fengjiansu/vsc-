% Created 2019-04-23 Tue 16:54
\documentclass[11pt]{article}
\usepackage[utf8]{inputenc}
\usepackage[T1]{fontenc}
\usepackage{fixltx2e}
\usepackage{graphicx}
\usepackage{longtable}
\usepackage{float}
\usepackage{wrapfig}
\usepackage{rotating}
\usepackage[normalem]{ulem}
\usepackage{amsmath}
\usepackage{textcomp}
\usepackage{marvosym}
\usepackage{wasysym}
\usepackage{amssymb}
\usepackage{hyperref}
\usepackage[UTF8, heading = false, scheme = plain]{ctex}
\tolerance=1000
\author{宋昕}
\date{\today}
\title{lab1}
\hypersetup{
  pdfkeywords={},
  pdfsubject={},
  pdfcreator={Emacs 25.1.1 (Org mode 8.2.10)}}
\begin{document}

\maketitle
\tableofcontents

\section{lab1 分治法实验}
\label{sec-1}
\subsection{任务一}
\label{sec-1-1}
编写BubbleSort.c和MergeSort.c,利用提供的数据集,分别记录不同大小数组的排序时间,并根据规模和时间画出两个算法的实际测量时间复杂度。
\subsection{任务二 (计算机算法设计与分析,第五版) P41开始,算法实现题目}
\label{sec-1-2}
\subsubsection{2-1 众数问题(要求用分治法,即递归的方法完成)}
\label{sec-1-2-1}
第一步:对集合(数组)中的元素进行排序
第二步:设计一个递归函数Mode,输入参数是排序过的一维数组以及数组的起点和终点,返回值是一维数组的众数
第三步:设计main函数,并测试
\begin{verbatim}
int Mode(int a[], int i, int j) {
  if (i == j) return 1;
  
  int m = (i + j) / 2;


}
\end{verbatim}
\subsubsection{2-3 半数集问题}
\label{sec-1-2-2}
第一步:理解问题,用小规模输入进行模拟演算,确保理解题意
第二步:进行递归式思考,分析问题的输入和输出,原问题的输入/出是什么,子问题的输入/出是什么,以及原问题分解为几个子问题,基本情况的条件和值是什么?
第三步:尝试写出计算半数集的递推式,递推式中的多个子问题会用到求和公式Σ,要明确求和的上、下边界
第四步:把递推公式改写为程序代码HalfSet.c

\begin{verbatim}
cat HalfSet.c
\end{verbatim}

\subsubsection{2-5 有重复元素的排列问题}
\label{sec-1-2-3}
第一步:与无重复元素的排列问题作比较,找到重复发生的环节,然后设置去重的条件
第二步:编码实现
\begin{verbatim}
cat PermRemove.c
\end{verbatim}
\subsubsection{2-6 字典序问题}
\label{sec-1-2-4}
    第一步:观察无重复全排列的次序(用课堂上的讲解的算法)和字典序次序的差异,进而发现字典序和的差异在于从序列中选出一个打头元素后,当前打头元素该放在什么位置?课堂算法的做法是当前打头元素和选出的新打头元素交换位置,而字典序是把选出的打头元素放在当前打头元素之前,选出打头元素后,其空下的位置由后续元素向前补充。因此需要修改全排列生成算法中的Swap功能。
第二步:编码实现第一步的构思
\begin{verbatim}
cat DictOrder.c
\end{verbatim}
第三步:分析推算2, 6, 4, 5, 8, 1, 7, 3序列的次序(应为8228),从推算过程中看看能否写成一个递归式:\[ f(i, n, seq) = f(i + 1, n, seq) + (order(A[i], seq) - 1) * (n - 1)!\], if \[i + 1 = n\] ,then \[f(i, n, seq) = 1 \]
\subsubsection{2-7 集合划分问题(1)}
\label{sec-1-2-5}

\subsubsection{2-8 集合划分问题(2)}
\label{sec-1-2-6}
\subsubsection{2-9 双色汉诺塔问题}
\label{sec-1-2-7}
双色汉诺塔问题和单色汉诺塔问题在输入规模一样的情况下,挪动盘子的次数是一样的。但过程却是不同的,单色汉诺塔有两个子问题,双色汉诺塔有四个子问题,双色公式\[\]
\subsubsection{2-11 整数因子分解问题}
\label{sec-1-2-8}
% Emacs 25.1.1 (Org mode 8.2.10)
\end{document}